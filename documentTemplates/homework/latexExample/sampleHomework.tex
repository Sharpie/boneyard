% !TEX TS-program = pdflatex   or Sweave

\documentclass{homework}
%\documentclass[Sweave]{homework}

%---------------------------------------------------------------------
% Sweave setup
%---------------------------------------------------------------------
\ifSweave
	\usepackage{Sweave}
	\newcommand{\sweaveDateStamp}{\texttt\Sexpr{Sys.time()}}
\fi

%---------------------------------------------------------------------
% Homework information
%---------------------------------------------------------------------
\newcommand{\hmwkTitle}{Another Assignment}
\newcommand{\hmwkDueDate}{\today}
\newcommand{\hmwkClass}{ENGR 666: Your Former Social Life}
\newcommand{\hmwkAuthorName}{John Doe}


%---------------------------------------------------------------------
% Font selections
%---------------------------------------------------------------------
\usepackage{mathpazo}


%---------------------------------------------------------------------
% Additional Packages
%---------------------------------------------------------------------

% Support for pretty-printing numbers and units.
\usepackage{siunitx}

% Provides pretty-printing of fractions like 1/2 that look nice when used in-line
% with text.  This package also tries to be aware of and adjust for font choices.
\usepackage{xfrac}

%---------------------------------------------------------------------
% Layout tweaks
%---------------------------------------------------------------------


%---------------------------------------------------------------------
% Macros (Shortcuts)
%---------------------------------------------------------------------

% Derivatives and partial derivatives
\newcommand{\deriv}[3][]{\ensuremath{\frac{\mathrm{d}^{#1} #2}{\mathrm{d} #3^{#1}}}}
\newcommand{\pderiv}[3][]{\ensuremath{\frac{\partial ^{#1} #2}{\partial #3^{#1}}}}


%---------------------------------------------------------------------
% The main event
%---------------------------------------------------------------------
\begin{document}
\maketitle
%---------------------------------------------------------------------

\begin{homeworkProblem}
 
\SI{1}{\milli\liter} of dye with a density of \SI{1.0}{\gram\per\milli\liter} is spread evenly across the top of water in a glass containing water with a density of \SI{1.0}{\gram\per\cubic\centi\meter}.  The glass has a radius of \SI{4}{\centi\meter} and depth of \SI{15}{\centi\meter}.  Assuming the dye has a typical liquid diffusion coefficient of \SI{1e-5}{\centi\meter\squared\per\second}, how long will it take for the dye concentration at the bottom of the glass to reach \sfrac{1}{10} of the equilibrium concentration?  Assume the dye immediately and uniformly spreads out across the top of the glass.

\begin{homeworkSection}{a)}{How many dimensions would you use to model this system?}

\ansBox{
I would assume that the dye disperses uniformly in the radial direction such that the concentrations in $x$ and $y$ are well mixed.  I would also assume that the walls of the container do not introduce any boundary effects.  Based on these assumptions, I would neglect cocentration changes in the $x$ and $y$ directions and the only direction I would worry about would be the $z$ direction or depth of the glass.
}

\end{homeworkSection}

\begin{homeworkSection}{b)}{Where would you locate the zero of the coordinate system?}

\ansBox{
I would place the zero point of the $z$ axis at the surface of the water and define positive as a direction pointing into the glass. The positive values of the $z$ coordinate would therefore represent depth below the surface.
}

\end{homeworkSection}

\begin{homeworkSection}{c)}{What is the governing mass balance equation for this scenario? Include only terms that are relevant.}

\ansBox{
Assuming dispersion is the only significant transport mechanism at work, the governing equation would be:

\[
\pderiv{C\left( z,t\right)}{t} = D_{z}\pderiv[2]{C\left( z,t\right)}{z}
\]

}

\end{homeworkSection}

\begin{homeworkSection}{d)}{What is a boundary condition you will have to deal with? Define all distances relative to the coordinate system.}

\ansBox{
A boundary condition I would impose is that the gradient of concentration across the bottom of the glass is zero. So, if $z_{D}$ represents the depth of the glass:

\[
\left.\deriv{C}{z}\right|_{z=z_{D}} = 0\quad\forall\ t
\]

}

\end{homeworkSection}


\end{homeworkProblem}

%---------------------------------------------------------------------
\end{document}
%---------------------------------------------------------------------
