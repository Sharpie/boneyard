% !TEX TS-program = pdflatex
\RequirePackage[l2tabu, orthodox]{nag}
\documentclass{homework}


%---------------------------------------------------------------------
% Sweave setup
%---------------------------------------------------------------------
\ifSweave
	\usepackage{Sweave}
	\newcommand{\sweaveDateStamp}{\texttt\Sexpr{Sys.time()}}
\fi

%---------------------------------------------------------------------
% Homework information
%---------------------------------------------------------------------
\newcommand{\hmwkTitle}{Another Assignment}
\newcommand{\hmwkDueDate}{some day}
\newcommand{\hmwkClass}{For some class}
\newcommand{\hmwkAuthorName}{John Doe}


%---------------------------------------------------------------------
% Font selections
%---------------------------------------------------------------------
\usepackage{mathpazo}


%---------------------------------------------------------------------
% Additional Packages
%---------------------------------------------------------------------

% Support for pretty-printing numbers and units.
\usepackage{siunitx}

% Provides pretty-printing of fractions like 1/2 that look nice when used in-line
% with text.  This package also tries to be aware of and adjust for font choices.
\usepackage{xfrac}

% For chemical formulae
\usepackage[version=3]{mhchem}

% Used for cross-referencing
\RequirePackage[pdftex,
	plainpages=false,
	pdfpagelabels,
	bookmarks,
	bookmarksnumbered,
	draft=false,
	breaklinks=true]{hyperref}

\RequirePackage{amsfonts,amsthm,amssymb}
\RequirePackage{mathrsfs}
\RequirePackage{amsmath}


%---------------------------------------------------------------------
% Layout tweaks
%---------------------------------------------------------------------


%---------------------------------------------------------------------
% Macros (Shortcuts)
%---------------------------------------------------------------------

% Derivatives and partial derivatives.
\newcommand{\deriv}[3][]{\ensuremath{\frac{\mathrm{d}^{#1} #2}{\mathrm{d} #3^{#1}}}}
\newcommand{\pderiv}[3][]{\ensuremath{\frac{\partial ^{#1} #2}{\partial #3^{#1}}}}

% Shortcuts for setting table column headers in different styles.
\newcommand{\colHead}[2]{
  \multicolumn{1}{>{\bfseries}#1}{#2}%
}
\newcommand{\ccolHead}[2]{%
  \multicolumn{1}{>{\centering\bfseries}m{#1}}{#2}%
}

% Physical units
\DeclareSIUnit{\atmosphere}{atm}

%---------------------------------------------------------------------
% The main event
%---------------------------------------------------------------------
\begin{document}
\maketitle
%---------------------------------------------------------------------


%---------------------------------------------------------------------
\begin{homeworkProblem}
% Begin problem 1
%---------------------------------------------------------------------
A dye is spread evenly across the top of a glass containing water.


\begin{homeworkSection}{What is the governing mass balance equation that describes how the concentration of dye changes throughout the glass as time passes? Include only terms that are relevant.}

\ansBox{
Assuming dispersion is the only significant transport mechanism at work, the governing equation would be:

\[
\pderiv{C\left( z,t\right)}{t} = D_{z}\pderiv[2]{C\left( z,t\right)}{z}
\]

}

\end{homeworkSection}


\begin{homeworkSection}{What is a boundary condition that must be applied to solve the governing equation?}

\ansBox{
A boundary condition that should be imposed is that the gradient of concentration across the bottom of the glass is zero. If $z_{D}$ represents the depth of the glass this condition can be expressed as:

\[
\left.\deriv{C}{z}\right|_{z=z_{D}} = 0\quad\forall\ t
\]

}

\end{homeworkSection}
%---------------------------------------------------------------------
\end{homeworkProblem}
% End problem 1
%---------------------------------------------------------------------




%---------------------------------------------------------------------
\begin{homeworkProblem}
% Begin problem 2
%---------------------------------------------------------------------
\newcommand{\napl}{{\scshape Napl}}

A gasoline spill whose constituents are given in \autoref{tab:breakdown} occurs on a soil whose bulk density, $\rho_{b}$, is \SI{1.6}{\kilo\gram\per\liter} and has an organic carbon content of 1\%. The table shows the mass of each constituent in the Non-Aqueous Phase Liquid (\napl ). Note the table lists the constituent concentrations in the \napl\ prior to partitioning. The porosity $n=0.4$ is divided among the water, \napl\ and air phases. The \napl\ phase constitutes 5\% of the volume and the air phase 20\%. 

Derive a set of equations that shows the partitioning, in mass concentration, among the the soil, water, \napl\ and air phases for an arbitrary constituent at equilibrium.

\begin{table}[!ht]

\caption{Breakdown of spill constituents}
\label{tab:breakdown}

\begin{tabular}{lSSSSS}
\toprule
\ccolHead{1in}{Constituent} & \ccolHead{1in}{Concentration \si{\gram\per\liter}} & 
\ccolHead{1in}{Molecular Weight \si{\gram\per\mole}} & \ccolHead{1in}{Henry's Law Constant \si{\atmosphere\liter\per\mole}} &
\ccolHead{0.5in}{$K_{\mathrm{oc}}$ \si{\liter\per\kilo\gram}} & \ccolHead{1in}{Fuel--Water Partition Coefficient ($K_{\mathrm{o}}$)} \tabularnewline
\midrule
\ce{C7H8} & 43.60 & 92.00 & 0.01 & 300 & 1250 \tabularnewline
\ce{C8H10} & 71.82 &106.00 & 0.01 & 830 & 3630 \tabularnewline
\bottomrule
\end{tabular}

\end{table}



\begin{homeworkSection}[Volume Breakdown]{}

The first task is to determine the fraction of a representative \SI{1}{\liter} volume that will be occupied by each constituent.  This can be accomplished using the porosity, $n$, and the information concerning the fraction of the void space occupied by the air, water and \napl\ phases:

\newcommand{\vol}[1]{V_{\mbox{#1}}}

\begin{align}
\label{eqn:volBreakdown}
\vol{soil} & = & (1-n)\cdot\vol{total} & = & \SI{0.6}{\liter} \nonumber\\
\vol{air} & = & 0.20\cdot n\cdot\vol{total} & = & \SI{0.08}{\liter} \nonumber\\
\vol{\napl} & = & 0.05\cdot n\cdot\vol{total} & = & \SI{0.02}{\liter} \nonumber\\
\vol{water} & = & 0.75\cdot n\cdot\vol{total} & = & \SI{0.3}{\liter}
\end{align}

\end{homeworkSection}


%---------------------------------------------------------------------
\end{homeworkProblem}
% End problem 2
%---------------------------------------------------------------------


%---------------------------------------------------------------------
\end{document}
%---------------------------------------------------------------------
